%-------------------------------------------------------------------------------
%	SECTION TITLE
%-------------------------------------------------------------------------------
\cvsection{Articles de Blog}


%-------------------------------------------------------------------------------
%	CONTENT
%-------------------------------------------------------------------------------
\begin{cventries}

%---------------------------------------------------------
  \cventry
    {Auteur} % Role
    {Cartographie avec des outils open source, Deuxième partie (2/4): Avec Python} % Title
    {knsamati.netlify.app} % Location
    {2020-04-16} % Date(s)
    {
      \begin{cvitems} % Description(s)
        \item {Deuxième partie de la serie d'articles consacrés à la cartographie avec les outils Open Source}
      \end{cvitems}
    }

%---------------------------------------------------------
  \cventry
    {Auteur} % Role
    {Manipulation des données volumineuses (Big Data) avec R} % Title
    {knsamati.netlify.app} % Location
    {2020-04-12} % Date(s)
    {
      \begin{cvitems} % Description(s)
        \item {Il s'agit d'une introduction à l'utlisation de R pour la manipulation des données volumineuses (Big Data) à travers les packages data.table, dask, etc.}
      \end{cvitems}
    }

%---------------------------------------------------------
 \cventry
    {Auteur} % Role
    {Introduction à l'utilisation des Bases de Données avec R et Python} % Title
    {knsamati.netlify.app} % Location
    {2020-02-05} % Date(s)
    {
      \begin{cvitems} % Description(s)
        \item {De plus en plus des données sont stockées sous forme de base de données. Avec R et Python, il est assez aisé d'interroger ces données et de pourvoir les utiliser avec les mêmes logiciels pour faire les analyses nécessaires. C'est l'objectif de cet article.}
      \end{cvitems}
    }

%---------------------------------------------------------
 \cventry
    {Auteur} % Role
    {Cartographie avec des outils open source. Premiere partie (1/4) : Avec R} % Title
    {knsamati.netlify.app} % Location
    {2020-01-05} % Date(s)
    {
      \begin{cvitems} % Description(s)
        \item {La première partie de la serie de billets consacrés à l'utilisation des outils Open Source pour la cartographie}
      \end{cvitems}
    }

%---------------------------------------------------------
 \cventry
    {Auteur} % Role
    {Utiliser R et Python pour se connecter à l'API de kobotoolbox (ODK) afin d'automatiser ces resultats} % Title
    {knsamati.netlify.app} % Location
    {2020-01-03} % Date(s)
    {
      \begin{cvitems} % Description(s)
        \item {L'utilisation de ODK et les logiciels Open Source tels que R et Python, permet d'automatiser la collecte des données et la diffusion des données. C'est l'objectif de cet article qui utilise l'API de kobotoolbox pour se connecter aux données et de pour l'utiliser dans les outils de tableau de bord comme Shiny(R) et Dash (Python)}
      \end{cvitems}
    }

%---------------------------------------------------------
\end{cventries}
